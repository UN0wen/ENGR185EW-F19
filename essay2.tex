\documentclass[12pt]{article}

\usepackage[margin=1in]{geometry}
\usepackage{graphicx}
\usepackage{placeins}
\usepackage{titlesec}
\usepackage{float}
\usepackage[english]{babel}
\usepackage[backend=biber, style=apa]{biblatex}
\addbibresource{essay2.bib}
\titleformat*{\section}{\large\bfseries}
\begin{document}
		\title{Ethical Case Study: \\Texas City Refinery Explosion}

	\author{Duy Duong\\UID: 505183737\\Discussion 1B\\3000 words}
	\maketitle
	
	\begin{abstract}
Placeholder
	\end{abstract}
	\setlength{\parskip}{1em}
	\section*{Problem statement}
	The world runs on oil. Oil refining is an essential step to make the crude oil extracted from the ground usable in our daily lives, but the lack of any real innovation in the oil refining process in recent years meant that oil refineries sometimes still run on equipment decades old, prone to becoming a hazard. Accidents at these plants are common; as a consequence, their safety codes have only gotten stricter over the years. However, some plants still slip through the cracks and let truly disastrous accidents unfold: In 2005, an oil refinery in Texas City exploded, causing catastrophic damages both to the plant and the surrounding area as well as tragic loss of human life. The incident was a result of years of mismanagement as well as a lack of due diligence on the part of the engineers running the day to day operation. In order to stop similar accidents from happening in the future, we must utilize ethical analysis to find out the deciding factors that led to this disaster. From this information, 
	\section*{Background}
	
	On March 23, 2005, an oil refinery in Texas City, owned by British Petroleum (BP), exploded and killed 15 people, injuring 170 more. The plant was built in 1934 and acquired by BP in 1999, at which point it had already fallen out of use for many years, with many crucial safety upgrades postponed by its previous owner (\cite{frontline_2010}). Part of maintenance work BP planned to do with the plant include an isomerization unit (ISOM), which converted low octane hydrocarbons (organic compounds of hydrogen and carbon) into high octane ones that can be blended into unleaded gasoline. This unit contained a raffinate splitter - a tower that separates light hydrocarbon components from heavy ones (such as octane), shown in Figure \ref{fig:ISOMUnit}. After the hydrocarbons are split, they travel in 3 directions: the heavy and light hydrocarbons are pumped to the heavy and light raff storage tank respectively, and the hot vapors travel to the blowdown drum, which acted as a cooler and waste disposal unit for these gases.
		
	\begin{figure}[H]
		
		\includegraphics[width=\textwidth]{BP_Texas_City_incident_diagram.png}
		\caption{Incident diagram of the explosion. Compared to the normal flow diagram, the flow to the heavy storage tank was blocked, and raffinate feed overflowed through the top to the blowdown drum. (\cite{BPreport}})
		\label{fig:ISOMUnit}
	\end{figure}

	The raffinate splitter, in normal operation, would contain around 2m of liquid from its base. This liquid level was controlled by many overlapping safety systems, all of which had one or more major issues that were not acknowledged nor resolved. On the day of the accident, the heavy storage tank was nearly full from the previous night's operation and the level control valve was shut off, cutting the downward flow to the heavy tank. When the raffine splitter was filled the next morning, the liquid level went up many times the safe amount as there was nowhere else for the materials to go. The raffinate splitter tower overflowed when the liquids expand due to the heat during operation, sending liquid hydrocarbons to the blowdown drum, as seen in Figure \ref{fig:ISOMUnit}. As the blowdown drum inevitably overflowed, hot liquid hydrocarbons shot off from the top and flowed through the drain pipes for sewer disposal.

	In the immediate vicinity were workers working on other parts of the ISOM, with their trailers parking right beside the plant, less than 110m away. The closest trailer was only 45m away from the blowdown drum. When the hot liquid hydrocarbons came out of the blowdown drum, the resulting vapor cloud consisting of highly flammable hydrocarbon fumes caused an idle truck engine to overheat and sparked a fire, igniting the vapor cloud. 
	
	The result was a massive explosion that completely destroyed most of the trailers, killing 15 instantly and injuring 180 more, mostly in the trailer areas. The explosion shattered windows three quarters of a mile away and damaged tanks holding hazardous materials such as benzene, leaking them to the environment. The subsequent fire burned up to 19000 $m^2$ of the refinery, causing damages in the millions of dollars. The ISOM itself did not return to normal operation until two years later after sustaining severe damages in the fire. (\cite{csbreport})
	
	The disaster's impact is still felt years after the accident. A study conducted during and after the incident showed that it had led to a decline in the perceived mental and physical health of the people living in the area (\cite{Peek106}J). Lawsuits from the victims' families have followed BP, and they have had to pay out 1.6 billion USD in compensation. One notable lawsuit was Eva Rowe's, who demanded justice be brought on the company and forced them to release internal documents showing that they had prior knowledge of the issues plaguing the ISOM. These internal reports later proved to be vital in bringing to light the repeated failures that eventually led to the fatal incident.

	\section*{Engineering failure}
	In 2002, three years before the disaster, Texas City refinery's director ordered an internal report on the mechanical integrity and safety of the plant in response to its obvious decline in the years after the plant's acquisition in 1999. The study concluded that mechanical integrity was an enormous issue, and there were many vulnerabilities in the plant's infrastructure. It then proposed a budget increase of 235 million dollars to rectify these issues, one goal that was eventually not met (\cite{csbreport}) Many audits and studies followed, but no significant action was taken to conduct maintenance work on the plant, mainly due to continued budget cuts in 2003 and 2005. BP justified these budget cuts as "it had not made a contribution of profit proportionate to its capital consumption," while ignoring that the plant could not operate to its maximum capacity because of its aging and failing infrastructure.
	
	Made two months before the incident but only unveiled a year after the Texas City refinery explosion, another internal report commissioned by BP following other accidents happening in 2004 further detailed the multitudes of existing issues with the plant. This report, called the Telos Report after its author The Telos Group, assessed the "safety behavior and culture" at the plant, specifically the plant's leadership, culture, and safety management. The report displayed how production and budget compliance was valued above all else, and engineers were put under immense pressure to produce while being understaffed and lacking adequate training. It also found various safety issues present on site: alarms were broken, concrete was falling off, and fumes were enveloping the plant. As its final verdict, the Telos Report concluded: “We have never seen a site where the notion ‘I could die today’ was so real” (\cite{telosreport}).
	
	 The compounding issues hounding the plant carried out a major role in the accident. The raffinate splitter had just been completed, and the start-up process of the plant went into action the night of March 22. One crucial part of this process was the BP Pre-Startup Safety Review procedure consisting of thorough technical and personnel checks, which was not carried out possibly due to a lack of training. The start-up process began with the Night Operator filling the raffinate splitter. Operation through the night meant led to the heavy storage tank being filled completely the morning of March 23 and the flow from the raffinate tower was shut off, but this information was not relayed to the operators working later shifts, once again showing negligence on adhering to safety protocols by the operators involved. 
	 
	  Many of the safety equipments in the raffinate tower were also defective, playing a major part in the accident's progress. When the raffinate splitter tower was refilled in the morning, it overfilled many times its capacity, but this was not caught by any of the engineers involved as the defective level indicators did not give any warning, and the sight glass for manual checking was broken.  Another system that could have prevented the issue was the level alarm in the blowdown drum that did not go off even though hydrocarbons were bursting out of its top. 
	  
	  The last egregious issue was a managerial one: the location of the trailers in the plants led to an unnecessary loss of life. In 2002, a site wide temporary siting analysis laid out the allowed parking spaces for trailers. However, work on an adjacent building meant that extra trailers were placed next to the ISOM in 2004, despite the team facilitating this change not having the required knowledge to assess the risk to the trailers in an event of an explosion. In the accident, 12 of the 15 fatalities were contractors inside the trailer situated closest to the ISOM, while the other 8 occupants of that trailer were seriously injured. All of the deaths came from blunt force trauma caused by the equipment in the trailers when the explosion sent a massive shockwave outwards. (transition?)
	
	\section*{Ethical analysis}
	These engineering failures, in retrospective, were all preventable, yet nothing was done. Through ethical analysis, we can inspect the actions of the involved parties with an ethical framework to  pinpoint the moral failures that led to these devastating consequences and prevent these decisions from repeating. Deontology, or duty ethics, is one of these frameworks: this approach to normative ethics states that actions are morally right as long as they follow a specific set of moral rules. These rules, also known as maxims, must follow two core principles - the universality principle and the reciprocity principle. Based on these maxims, we can categorically define each action as right or wrong independent of their consequences.
	
	The first core principle of duty ethics, the universality principle states that a maxim should only be followed if it can be turned into an universal law. In other words, the maxim must be universally good such that everyone would agree to following it while not ending up contradicting itself. Since these laws are unambiguous and straightforward, however, sometimes they come into conflict with each other. In these cases, the maxims are further divided into prima facie norms (what initially seems to be good) and self-evident norms (what is good when we consider the full context of the situation), where the self-evident norms are usually followed. The second principle in duty ethics is the reciprocity principle, which describes the need for one to respect the rationality of other humans. A maxim can only follow this principle if they do not simply treat humans as a tool to be used for an end. These two principles are the underlying pillars of duty ethics, and actions not following maxims that adhere to these two principles can be considered unethical.
	
	Applied to the Texas City refinery case, deontology can shed light on the morality behind actions taken before, during, and after the incident. The first immediately obvious ethical failure was the inaction of BP and its directors in spite of the internal audits and reports asking for change. Instead of carrying out the required safety upgrade and maintenance work, the company overvalued the importance of production and cited the plant's lack of profits as one of the reason to cut its budget, further undermining its capability to move on from aging equipment (\cite{csbreport}). Here, BP followed the maxim: "It is unnecessary to follow safety advice from internal reports if it doesn't bring immediate profits." This maxim fails the universality principle, as it is clear that following safety codes is almost always detrimental to a plant's profits as they add more training for their engineers and more redundancy to their core systems. Therefore, internal reports suggesting these safety upgrades would always be ignored, and they would become pointless because no change would ever happen. If this maxim becomes an universal law, no internal investigations would ever be made again, creating a contradiction. BP's inaction was morally wrong, and the incident would have been entirely preventable had the engineers have up to date equipment to let them know what was going wrong.
	
	Another factor contributing to the disaster's catastrophic damages was the irresponsible approval of trailer parking spaces near the ISOM, which directly led to most of the deaths and injuries in the explosion. This approval was also an ethical failure: management did not disclose their lack of knowledge to assess the risks associated with the trailer placements to the workers living in those trailers, and went against procedure to accelerate the approval process (\cite{csbreport}). The maxims the plant management followed was that of "Contract workers don't need to know the full details of their work situation" and "Accelerating work is more important than following required procedures." These former maxim fails the reciprocity principle as it treats the contracted workers as mere tools to an end. Lacking knowledge of the potential dangers associated with their trailer placements, the workers were stripped of the choice to report the issue or potentially leaving the site. The latter maxim - "Accelerating work is more important than following required procedures"  -  also fails to follow one of the core principles of deontology, this time the universality principle. Since these procedures are created to add multiple layers of supervision as well as engineering input to ensure that there are no critical errors in the final decision, following them is always slower than just directly making decisions without thinking of the consequences. Therefore, universalizing this maxim means that nobody would follow due process anymore, completely eliminating the purpose of these procedures and causing a contradiction. Following duty ethics, the management's hasty and irresponsible decision was morally condemnable, and the explosion would have been much less severe had they follow due process and not rush to decisions outside of their expertise.
	
	The final deciding element that led to this disaster was the negligence displayed by the engineers in charge of the startup sequence - a damning display of the company's lax safety culture. Even though BP had a crucial safety procedure called BP Pre-Startup Safety Review that was supposed to be mandatory, the operators in charge did not follow them. This led them to believe the readings provided by the defective sensors without question, and eventually led to the overfilling of the tower and the explosion itself. Leaving aside the obvious ethical failure in them not adhering to safety laws, the engineers also acted on the maxim "follow normal operating procedures when there is no alarm." Although seemingly rational at first, when taking the context into consideration, it is clear that this maxim was just a prima facie norm as there was reasonable cause for safety concerns: there had been no pre-startup review and the Night and Day operators did not communicate directly. Instead, the engineers should have acted according to the self-evident norm "safety is the most important thing" and manually confirmed that the raffine splitter tower was functioning correctly as well as establishing communication between teams to rule out any potential safety issues.
	
	\section*{Recommendations}
	There were a multitude of actions that should have been done to stop the disaster from happening, all adhering to deontological principles. For a start, BP should have listened to their engineers' recommendations presented in the internal reports spanning years before the accident. Had their concerns been heard, the plant's engineers would have had the budget needed to update its aging infrastructure. The working alarm systems would have alerted operators to an anomaly in the plant's operation and gave them ample time to resolve the issue. This action on BP's part would have followed the maxim "safety before profits," which is easily universalizable . 
	
	Additionally, the lax safety culture at BP played a major role in the accident, so further training should have been conducted to enforce the safety first mindset across their employees. Engineers should have been heavily penalized for not following safety procedures, and management should not have been able to make sweeping changes without a complete understanding of the consequences through an external review process. Otherwise, the workers should have been informed of the directors' lack of knowledge regarding their trailer placement. This action would have followed the reciprocity principle of deontology as it treats the workers no longer as a mere tool but rational beings who can form their own decisions. 
	
	The safety issues were further compounded by a lack of supervision by governmental agencies, specifically OSHA (Occupational Safety and Health Administration), who continued to let the plant operate despite its obvious safety failings. To prevent similar accidents from happening in the future, there must be stricter safety requirements for oil refineries and routine inspections to ensure the application of these requirements across the nation. Furthermore, plants with bad track records such as the Texas City refinery must be heavily scrutinized and forced to stop operating after any accident until they have shown clear improvements, which would minimize the chances of similar situations from happening in the future. If this regulation had been in place before the Texas City explosion, the refinery would have been forced to renovate after its various failures in the years beforehand, stopping the disaster from ever happening.
	\section*{Conclusion}
	The Texas City refinery explosion was a foreseeable and preventable event. Lives were lost because of egregious ethical lapses on the part of multiple parties, and the accident would not have happened were any of the involved parties adhere to the core principles of deontology. In the years since, many new regulations aiming to improve safety in oil refineries have been introduced (\cite{calaw}), although only to limited success: In 2015, almost the exact same situation played out in an ExxonMobil oil refinery in Torrance, CA. Due to a lack of oversight, aging equipment in this refinery degraded and caused an explosion that sent thousands of pounds of hydrofluoric acid into the surrounding area (\cite{exxonmobilreport}). It is sufficient to say that our attempts to ensure the safety of oil refineries so far have been ineffective - the inherently dangerous nature of fossil fuels being stored in confined areas with high heat is still too big of a blockade for us to create any meaningful attempts forward. It is perhaps in our best interest to stop holding onto the relics of the past; forcing the closure or renovation of refineries that are too old might be costly in the short term, but it might be our best bet yet to prevent the unnecessary loss of lives from these accidents.
	
	\printbibliography
	%\section*{Sources}
	% https://www.pbs.org/wgbh/pages/frontline/the-spill/bp-troubled-past/
	% https://www.csb.gov/bp-america-refinery-explosion/
	%https://people.uvawise.edu/pww8y/Supplement/-ConceptsSup/Work/WkAccidents/BPTxCityFinalReport.pdf
	%https://www.nytimes.com/2010/07/13/business/energy-environment/13bprisk.html?pagewanted=all
	%https://www.propublica.org/article/blast-at-bp-texas-refinery-in-05-foreshadowed-gulf-disaster
	%https://bloximages.newyork1.vip.townnews.com/galvnews.com/content/tncms/assets/v3/editorial/6/cb/6cb6d830-d239-11e4-9afa-2f2f90912f92/5511818abc022.pdf.pdf
 %	https://www.nytimes.com/2009/10/31/business/31labor.html 
	% https://jech.bmj.com/content/62/2/106
	%https://calepa.ca.gov/new-regulations-improve-safety-at-oil-refineries/

\end{document}