 \documentclass[12pt]{article}
 \usepackage{natbib}
 
 \usepackage[margin=1in]{geometry}
 \makeatletter
 \renewenvironment{thebibliography}[1]
 {\section*{\refname}%
 	\@mkboth{\MakeUppercase\refname}{\MakeUppercase\refname}%
 	\list{\@biblabel{\@arabic\c@enumiv}}%
 	{\settowidth\labelwidth{\@biblabel{#1}}%
 		\leftmargin\labelwidth
 		\advance\leftmargin20pt% change 20 pt according to your needs
 		\advance\leftmargin\labelsep
 		\setlength\itemindent{-20pt}% change using the inverse of the length used before
 		\@openbib@code
 		\usecounter{enumiv}%
 		\let\p@enumiv\@empty
 		\renewcommand\theenumiv{\@arabic\c@enumiv}}%
 	\sloppy
 	\clubpenalty4000
 	\@clubpenalty \clubpenalty
 	\widowpenalty4000%
 	\sfcode`\.\@m}
 {\def\@noitemerr
 	{\@latex@warning{Empty `thebibliography' environment}}%
 	\endlist}
 \renewcommand\newblock{\hskip .11em\@plus.33em\@minus.07em}
 \makeatother
 

 \begin{document}
	\title{Placeholder}
	\setlength{\parindent}{0pt}
	\begin{flushleft}
		Duy Duong\\UID: 505183737\\Discussion 1B\\2400 words
	\end{flushleft}
	
	\hfill
	\begin{center}
		{\Huge Nitrate pollution of global water sources through agricultural fertilizer use}
	\end{center}
	\hfill
	 \setlength{\parindent}{4em}
	\setlength{\parskip}{1em}
	
	Mankind has always been dependent on the land and the environment that we live in, whether for gathering and hunting or for cultivating crops. As the years pass, we have found even better ways to maximize crop yields, notably with the introduction of nitrogen-based fertilizers . However, the excessive overreliance on nitrogen-based fertilizers and other nitrogen-containing products in agricultural areas has had its cost: the nitrogen present in these products will, over time, break down into nitrates and mix with groundwater, contaminating it and various other water sources as well. The continually increasing amount of nitrogen in groundwater, surface water, and the oceans has been associated with developing diseases upon human consumption as well as facilitating algae growth, which in turn uses up oxygen that other aquatic lifeforms needs to function \cite{skipton08}. Needless to say, finding the answer to this ever growing problem is no easy task. As the agriculture sector continues to rely on the effectiveness of nitrogen-based products, the potential political and economic costs of a solution to remove these products rises accordingly. Through his paper \textit{Tragedy of the Commons}, Garret Hardin provided us with a framework we can model our problem upon and an effective solution that we can adapt to our own problem. In particular, we must strike a balance between a technological and political solution by first finding effective ways to reduce nitrogen emissions from these products through more effective and less polluting fertilizers. Furthermore, we would need to regulate their use with laws limiting the maximum usage rates and penalizing overuse to avoid further contamination of our shared resources. 
	
	Through The Tragedy of the Commons, Hardin provides us with some key ideas with which we can describe the problem with nitrogen-based products overuse. The commons, as described by Hardin, is a finite, depletable shared resource that can be freely accessed and used by the general public. These commons are further classified based on how we can impact them: source type commons that are depleted by the extraction of resources and sink type commons that are destroyed through excessive pollution. Maintaining these commons is a matter of public interest as it provides continual benefits to the collective sharing its resources. Nevertheless, in the eyes of rational men acting in their best interests, the immediate individual profit gained from exploiting these shared resources far outweighs the damage shared uniformly among society. These conflicting interests come to a head as the tragedy of the commons, where Hardin asserts that this self-interest focused rationale will inevitably destroy the commons and that there are no purely technical or nontechnical solution to this problem. With his problem of overpopulation, it is abundantly clear that no purely technical solution exists. Hardin also explores other solutions such as appealing to conscience, which, shown by a simple Darwinian argument, surely leads to an “elimination of conscience from the race” (Hardin, 1968). The solution to this kind of problem, as Hardin argues, is mutual coercion: regulations against exploitation and promoting moderation in the use of the commons that are accepted by the citizen as necessary. 
	
	Using The Tragedy of the Commons as a framework, we can see that the groundwater and by extension all water sources are considered the commons. Specifically, they are sink type commons as even though they are finite resources that we draw from, the natural water cycle means that we would have a long way to go until we can deplete all water sources on Earth, making the pollution of these sources a more important factor in their destruction. As a species that rely on water to survive, we have a shared interest in keeping our water sources clean and uncontaminated not only for our own health interests but also for the production of essential food products that can be negatively impacted by water contamination. Increasing nitrate levels in water has been linked to an increased risk of birth defects and cancers upon human consumption as well as birds, corals, and fish diseases \cite{issuesinecology}. 
	
	There is no denying the importance of synthetic nitrogen in most common fertilizers. As plant growth is mostly limited by nitrogen deficiency associated with decreasing soil fertility, the addition of nitrogen is essential to ensure maximum growth rate. In fact, it is estimated that over 40\% of the world population in 1991 only survives because of the existence of nitrogen-based fertilizer increasing crop yields \cite{soilnitrogen}. However, the effectiveness of the additive nitrogen is highly dependent on the crop’s nitrogen use efficiency. The largest consumer of nitrogen, cereals, only has a nitrogen use efficiency of 33\%, which means that two thirds of the nitrogen added to the ground in fertilizers are left unutilized \cite{ussiri12}. The excess nitrogen then, due to runoff or various other reasons, leaks into the groundwater as nitrites or turn into nitrous oxide, a greenhouse gas. Even though this has no immediate effect, over time the buildup of nitrates in groundwater eventually leads to the contamination of most of our water sources - one third of U.S. streams and two fifths of U.S. lakes are found to have high nitrogen concentrations \cite{issuesinecology}. Similar issues with water quality has also been reported globally, especially in areas with high agricultural activity. Despite the clear benefits of nitrogen-based fertilizers, there is no denying the accumulating harmful effects it has on our water sources - an issue worsen by these fertilizers' overuse.
	
	 The self-interested rationale described by Hardin can be seen in both individuals and businesses as they focus on the potential benefits of creating and using nitrogen-based products, especially fertilizers, for their own gains with complete disregard to others. For the individual farmers, using fertilizer is a straightforward decision that only benefits them. As the effects of nitrate pollution on our water sources can take a long time to manifest and are often much subtler than other sources of pollution, the layperson is often blind to its negative consequences. Therefore, the choice to not use fertilizer for your crops is akin to self-sabotage: without the extra yield and quality that comes with better plant growth, the environmentally conscious farmer will become less competitive and inevitably be forced to go bankrupt. Businesses, on the other hand, thrive on this dependence on nitrogen-based fertilizer. The producers of fertilizer, in an effort to generate more profits through driving their costs and prices down and outpricing their competition, make it easier than ever for farmers to buy and use fertilizer. In the years between 1945 and 1985, nitrogen fertilizer use has increased nearly twenty-fold to over 10000 millions of kilograms per year \cite{usgs}. It is clear where the tragedy of the commons lies: through their pursuit of financial gain, the users and makers of nitrogen-based fertilizers has come into direct conflict with our society's interest to preserve these water sources essential to life. Even though the consequences of nitrate contamination was not widespread knowledge until recently, the excessive use of these nitrogen containing products has made its mark on our current environmental situation.
	 
	A variety of technical solutions, defined by Hardin as solutions  that require no changes in values and morality,  already exists to great success, but their adoption rates are abysmal. Multiple approaches to this problem from different angles have been considered; there have been attempts to either remove nitrate compounds from surface water or develop new alternative fertilizers that are more stable and can release nitrogen slowly and efficiently \cite{fallahpour09}. Many of the current technical solutions to this problem have been proven to be effective in a practical setting, reducing nitrogen losses to the environment by 30 to 50\%. However, there are still many roadblocks preventing these solutions from being widely adopted. First of all, more than a third of the world’s crop production is meant for livestock, which transformed all but 30\% of the nitrates contained in these crops into manure \cite{issuesinecology}. Traditionally, this nitrogen is utilized by the next crop of plants as the manure is used as fertilizer. With the rise of increasingly specialized agricultural areas however, manure is simply removed from animal farms, and a majority of nitrogen contained is loss during this process. They then become a major pollutant that are not always covered by the existing technical solutions but are still a direct consequence of the widespread usage of fertilizer. Furthermore, the agricultural sector struggles to obtain funding to support the necessary infrastructure needed to implement these proven solutions, especially in countries where farming is becoming less important in the country's economy . This issue is compounded by the unwillingness of customers to tolerate price changes to food, which is unavoidable given no external support from governments. It is clear that, like Hardin argued, no purely technical solution can solve this problem. Widespread knowledge of the cost of maintaining the status quo is the bare minimum needed to facilitate the adoption of regulations necessary to popularize the current technical solutions. 
	
	To the contrary, nontechnical solutions such as legislation to limit the water pollution from nitrogen fertilizer has already been introduced and adopted in multiple parts of the world, but their success varies greatly. The best example is the E.U. Nitrates Directive in 1991 aimed to “prevent nitrates from agricultural sources polluting ground and surface waters and by promoting the use of good farming practices” \cite{nitratesdirective}. This legislation designates Nitrates Vulnerable Zones (NVZ) that are particularly vulnerable to nitrate pollution and forces them adopt an otherwise voluntary Code of Good Agricultural Practice. This includes measures limiting the use of nitrogen fertilizer and requiring infrastructure to be built for manure storage. In the years since, member states in the northwest regions of the EU (U.K., Ireland, etc.) has reported decreasing values of nitrogen surplus in soil of up to 20 kg N per hectare per year \cite{grisven12}. However, despite clear evidence to the contrary, over 70\% of farmers participating in a study in Ireland was still “unconvinced about the appropriateness of certain measures from a farm management, environmental and water quality perspective” \cite{buckley12}. Echoing the same sentiments are farmers in the Lombardy Plain, Italy, one of the areas most affected by nitrates. Despite 56\% of the plain being designated as NVZs, the nitrate concentrations in most regions have only been trending up in the past 11 years  \cite{mus19}, signifying a lack of willingness for a change of attitude towards nitrogen fertilizer despite strict regulation. 
	
	In order to tackle this problem on a global level, an emphasis on the education of the consequences of nitrogen fertilizer and the presence of existing technical solutions is our first and foremost priority. Leveraging the idea behind Hardin's proposed "mutual coercion," it is essential that the citizen, after understanding the significant impact of nitrogen fertilizers, recognizes the importance behind the proposed taxes and legislation changes and accepts them as a necessary evil. The urgency of this education is clear: the current lack of knowledge has directly led to the failure in widespread adoption of the environmentally conscious policies in Europe. The case of Germany just recently is a perfect example of the extent people will go to bypass the legislations that they don’t support. The fertilizer laws of Germany contains many exceptions and loopholes as the powerful forces in the agricultural sector lobby against the EU’s Nitrates Directives, leading to their nitrate pollution problem being highlighted multiple times by multiple organizations. Germany now faces billions of dollars in fines, shared by all taxpayers, and attempts to control nitrates level in surface water continues to be ineffective \cite{metaeeb}. With a better understanding of the potential dangers of nitrogen fertilizer, German farmers could have accepted and pressured the government into complying with the E.U.'s directives, saving themselves from the hefty fine and embarrassment on the world stage. Knowledge is not the whole answer to our problem, but it is the requisite first step towards a complete solution.
	
	After using scientific knowledge as a basis to introduce a paradigm shift in our public consciousness, we must then enact what Hardin refers to as “social arrangements that will keep [our water sources] from becoming the commons in the first place". These arrangements include harsh penalties for continuing to use an excessive amount of nitrogen fertilizers as well as subsidies for the implementation of viable technical solutions, whether to reduce waste or improve efficiency. One final area where government support is crucial is the identification of areas at higher risk of denitrification using currently available technology and enacting legislation concerning nitrogen fertilizer use specifically targeting these regions. Because the nitrogen problem is significantly worse in some areas due to heavy agricultural activity or the specific physical and geochemical properties of the land (easy runoff, low natural N-reduction,etc.), spatially targeted regulation applying to these regions are significantly more efficient and cost-effective, as shown by a study by the Institute of Environmental Science and Research in New Zealand \cite{sarris19}. This measure is similar to that of the E.U.'s NVZ designation and the forced Code of Good Agricultural Practice, but expanded to the rest of the world, including developed countries with a sizable agricultural sector such as the U.S. or China. Though no governmental body has the authority to command unified global action on this issue, public pressure stemming from increased knowledge of the issue is pivotal to coerce the independent countries to adopt similar regulations, saving our water sources from inevitable destruction.
	
		Based on the popularity of nitrogen-based fertilizers, the nitrate contamination problem is no doubt a global one with devastating current and potential consequences. An immediate solution consisting of a combination of education and strict localized regulation that expedites the adoption of existing technical solutions and penalizes continued pollution is essential to our planet's well-being. However, this solution in particular and this paper at large only covers developed countries with the resources to enact wide changes to an important part of the economy. For developing countries with smaller economies or ones that rely greatly on crop cultivation, the answer to this problem is much more complex as your average farmer relies on nitrogen fertilizer's benefits to feed their families. Looking further beyond, the same problem is apparent in other types of fertilizers as well: phosphorus and potassium runoff creates water contamination whose effect is even less well known than nitrate pollution because the responsible fertilizers are much less needed. It is our responsibility as sensible consumers to learn about these problems, educate others on them, and pressure lawmakers to put our environment first and foremost lest we face the total destruction of our water sources. 
\pagebreak	
	\begin{thebibliography}{99}
		\bibitem[(Alenxander and Smith, 1990)]{usgs}
		Alexander, R. B. and Smith, R. A., 1990,
		County-Level Estimates of Nitrogen and Phosphorus Fertilizer Use in the United States, 1945 to 1985;
		Reston, Virginia: U.S. Geological Survey Open-File Report 90-130, 
		1 p. [https://permanent.access.gpo.gov/lps27555/
		lps27555/pubs.usgs.gov/of/1990/ofr90130/report.html]
		\bibitem[(Buckley, 2012)]{buckley12}
		Buckley, C., 2012,
		Implementation of the EU Nitrates Directive in the Republic of Ireland — A view from the farm;
		Ecological Economics, v. 78, pp. 29-36.
		
		\bibitem[(Davidson, 2009)]{davidson09}
		Davidson, E. A., 2009,
		The contribution of manure and fertilizer nitrogen to atmospheric nitrous oxide since 1860;
		Nature Geoscience, v. 2, issue 9, pp. 659 - 662.
		
		\bibitem[(Davidson et al., 2012)]{issuesinecology}
		Davidson et al., 2012,
		Excess Nitrogen in the U.S. Environment: Trends, Risks, and Solutions;
		Issues in Ecology,
		issue 15.
		
		\bibitem[(European Union, 1991)]{nitratesdirective}
		The Council of the European Communities,
		Council Directive of 12 December 1991 concerning the protection of waters against pollution caused by nitrates from agricultural sources (91/676/EEC): European Union 
		[http://data.europa.eu/eli/dir/1991/676/2008-12-11]
		
		\bibitem[(Fallahpour, Fazeli, and Mirbagheri, 2009)]{fallahpour09}
		Fallahpour, M. R., Fazeli, M., and Mirbagheri, S. A., 2009,
		Application of Enhanced Methods of Phosphorus and Nitrogen Removal from Wastewater; 
		World Environmental and Water Resources Congress 2009,
		pp.1-9.
		
		\bibitem[(Lal and Steward, 2018)]{soilnitrogen}
		Lal, R. and Stewart, B.A., 2018,
		Soil Nitrogen Uses and Environmental Impacts: Boca Ranton, CRC Press,
		392 p.
		
		\bibitem[(Macintosh, 2018)]{metaeeb}
		Macintosh, E., June 21 2018,
		Germany faces billions in fines for breaking EU laws on nitrate pollution: META [EEB].
				
		\bibitem[(Musacchio et al., 2019)]{mus19}
		Musacchio et al., 2019,
		EU Nitrates Directive, from theory to practice: Environmental effectiveness and influence of regional governance on its performance;
		Ambio,  pp. 1-13.
		
		\bibitem[(Ng, Eheart, and Cai, 2009)]{ng09}
		Ng, T. L., Eheart, J. W., and Cai, X., 2009,
		Water Quality Effects of Varying Crop, Fertilizer, and Carbon Prices;
		World Environmental and Water Resources Congress 2009,
		pp.1-9.
		
		\bibitem[(Ni et al., 2011)]{ni11}
		Ni et al., 2011,
		Environmentally Friendly Slow-Release Nitrogen Fertilizer;
		Journal of Agricultural and Food Chemistry, v. 59, pp. 10169-10175.
		
		\bibitem[(Power and Schepers, 1989)]{power89}
		Power, J. F. and Schepers, J. S., 1989,
		Nitrate contamination of groundwater in North America;
		Agriculture, Ecosystems \& Environment, v.26, issues 3-4, 
		pp. 165-187.
		
		\bibitem[(Sarris et al., 2019)]{sarris19}
		Sarris et al., 2019,
		The effects of denitrification parameterization and potential benefits of spatially targeted regulation for the reduction of N-discharges from agriculture;
		Journal of Environmental Management, v. 247, pp. 299-312.
		
		\bibitem[(Skipton et al., 2008)]{skipton08}
		Skipton et al.,, 2008,
		Drinking Water: Nitrate-Nitrogen;
		NebGuide.
		
		\bibitem[(Ussiri and Lal, 2012)]{ussiri12}
		Ussiri, D. and Lal, R., 2012,
		The Role of Fertilizer Management in Mitigating Nitrous Oxide Emissions;
		Soil Emission of Nitrous Oxide and its Mitigation,
		pp. 315-346.
		
		\bibitem[(van Grisven et al., 2012)]{grisven12}
		van Grisven et al., 2012,
		Management, regulation and environmental impacts of nitrogen fertilization in northwestern Europe under the Nitrates Directive : a benchmark study;
		Biogeosciences, v. 9, issue 12, pp. 5143-5160.
		
		\bibitem[(Velthof et al., 2014)]{velthof14}
		Velthof et al., 2014,
		The impact of the Nitrates Directive on nitrogen emissions from agriculture in the EU-27 during 2000–2008;
		Science of The Total Environment, v. 468-469, pp. 1225-1233.
		
	\end{thebibliography}
 \end{document}

