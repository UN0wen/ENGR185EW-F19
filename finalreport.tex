\documentclass[12pt]{article}

\usepackage[margin=1in]{geometry}
\usepackage{graphicx}
\usepackage{tocloft}
\renewcommand{\cftsecleader}{\cftdotfill{\cftdotsep}}
\renewcommand{\cftsecfont}{\bfseries\normalsize}
\usepackage[utf8]{inputenc}
\usepackage{placeins}
\usepackage{amsmath}
\usepackage{titlesec}
\usepackage{float}
\usepackage{booktabs}% http://ctan.org/pkg/booktabs
\newcommand{\tabitem}{~~\llap{\textbullet}~~}
\usepackage{tabularx}
\usepackage{ltablex}
\renewcommand{\tabularxcolumn}[1]{>{\small\raggedright\arraybackslash}m{#1}}
\begin{document}
	\title{BottleGuard\\Final Report}
	\author{Duy Duong\\UID: 505183737 \and Alex Best\\UID: 404987130 \and Nayiree Dahyet \\ UID: 805361182 \and Samritha Nagesh \\UID: 505104679\and Jesus Martin \\UID: 605358449 \and Joshia Iglesia\\UID: 704980453 \and Timothy Yu\\UID: 304817925}
	\date{\today}
	\maketitle

	
	\section*{Proposal support}
	\begin{center}
			Name \underline{\hspace{8cm}}
		Signature \underline{\hspace{3cm}}\\ \vspace{5mm}
		Name \underline{\hspace{8cm}}
		Signature \underline{\hspace{3cm}}\\  \vspace{5mm}
		Name \underline{\hspace{8cm}}
		Signature \underline{\hspace{3cm}}\\  \vspace{5mm}
		Name \underline{\hspace{8cm}}
		Signature \underline{\hspace{3cm}}\\  \vspace{5mm}
		Name \underline{\hspace{8cm}}
		Signature \underline{\hspace{3cm}}\\  \vspace{5mm}
		Name \underline{\hspace{8cm}}
		Signature \underline{\hspace{3cm}}\\  \vspace{5mm}
		Name \underline{\hspace{8cm}}
		Signature \underline{\hspace{3cm}}\\  \vspace{5mm}
	\end{center}

\pagebreak
	\renewcommand*\contentsname{Table of Contents}
	\tableofcontents
		\vspace{10mm}
		
	\setlength{\parindent}{2em}
	\setlength{\parskip}{1em}
		\pagebreak
	\addcontentsline{toc}{section}{Abstract}
	\begin{abstract}
		While inhabiting shared living spaces like apartments have their advantages, such as cheaper living, they also have their fair share of annoyances, one of which being sharing beverage items.  Often times, it becomes hard  to effectively convey one’s preferences for what items should be shared and to what extent the sharing of that item should continue.  Additionally, it is hard to mitigate other individuals from simply disregarding another’s sharing preferences, and over indulging in their generosity.  The BottleGuard can assuage these problems by allowing the owner to communicate in real time to their apartment mates whether they want to share or not from anywhere in the world.  With a complimentary app, BottleGuard can be controlled remotely from anywhere with internet, signalling the owner’s sharing preference via a multi-colored led.  Users will also be notified of unwelcome use of their beverages as BottleGuard replaces the bottle’s original cap, and notifies owners of use with a gyroscope sensor built into the cap.  At a rather competitive price of \$25, 13,107 units will need to be sold in order to break even on the production and development costs.  Through proper production scheduling and adequate risk prevention and mitigation, BottleGuard will soon bring to customers the ability to share when desired while also maintaining the peace of mind that they won’t be taken advantage of.
	\end{abstract}

		\addcontentsline{toc}{section}{Introduction}
	\section*{Introduction}
	Bailey is a college student who lives in an apartment with several roommates. They all share the same fridge, and some of the items within. Although she is generally fine sharing her drinks with others, sometimes Bailey would like to keep them to herself when she gets a soda that is difficult to find or when the drink is running out. Due to their busy schedules, however, she finds it difficult to communicate that with all of her roommates. She resents the fact that sometimes her roommates help themselves to her drinks without asking for permission first because they were not aware that she didn't want to share. Bailey would like a way to remind her roommates in the moment of whether or not she is willing to share. Additionally, she wants to know when someone tries to open the bottle and take her soda when she has not granted permission.
	
	She has tried different methods of expressing her intentions, such as sticking Post-it notes on her bottles or sending a text message to their roommate group that she'd prefer if they didn't take her drink. However, these methods have not proven to be effective since text messages are often ignored or overlooked, and Post-it notes are prone to falling off the bottle. She finds a bottle lock called KwikTop but decides against using it since she understands that everyone’s schedule is different from hers, and such an inflexible method of keeping the drink to herself defeats the idea of sharing the drink, and can sour Bailey’s relationship with her roommates. Moreover, a bottle lock comes at a fixed size and can only fit on one type of bottle, while she wishes to share a variety of drinks that come in markedly different bottle sizes like soda bottles or milk jugs. IIdeally, she also wants something that does not take up much room in the fridge, as she only has a limited amount of fridge space allocated to her.
	
	Our device, BottleGuard, will solve all of Bailey’s problems, providing an advantage over competitors by allowing for a nearly instantaneous response notification to attempted use and to communicate the owner’s preferences to all users. BottleGuard is a compact cap replacement that can be fitted on to various types of bottles. It would let Bailey’s roommates know if she wanted to share the drink through a sharing mode that displays a green light to indicate a desire to share and a red light to indicate when one does not want to share. With a simple button press from the app, Bailey can set the light on the device from red to green from anywhere in the world with an Internet connection. The green light would then indicate her consent to share for the next couple minutes, allowing her roommate to pour their drink, before changing back to red again and signifying that permission is needed.  Bailey can also choose to keep the light green and leave sharing open. When the light is red, any attempt to open the bottle will trigger a warning sound and send a notification with the time the bottle was opened to Bailey’s phone. Bailey can now be more secure in knowing that she can be aware of if and when her roommates try to take her drinks, or use them without permission.
	
	There are a few risks inherent with BottleGuard, but nothing too dire.  The largest risks come from users trying to misuse the device by subjecting it to environments that BottleGuard was not designed for such as an oven or microwave.  These risks will be covered with more depth later on, however the greatest mitigation factor for a majority of the risks will be found in warning and instructions provided in a manual with each device.  BottleGuard is priced at a competitive twenty five dollars per device, allowing for the company to break even after the sale of approximately 13,107 units.  
	
		\addcontentsline{toc}{section}{Market Overview: Price / Performance}
	\section*{Market Overview: Price / Performance}
		\addcontentsline{toc}{subsection}{Price Performance Chart and Key Attribute Matrix}
	\subsection*{Price Performance Chart and Key Attribute Matrix}
	\begin{figure}[H]
		\includegraphics[width=\textwidth]{e185ew.png}
		\label{fig:ppc}
		\caption{Price Performance Chart}
	\end{figure}
\begin{figure}[H]
	\includegraphics[width=\textwidth]{kam.png}

	\caption{Key Attribute Matrix}
		\label{fig:kam}
\end{figure}
	\addcontentsline{toc}{subsection}{Competition}
	\subsection*{Competition}
	The top competitors for BottleGuard are KwikTop and a pack of sticky notes. Kwiktop is a combination lock that prevents access to the beverage without the combination. This product does not allow the user to easily share the beverage because the user locks the bottle, making it impossible to drink without the combination. If the user does not lock the bottle, another person may assume that the user wants to share, whether or not that is the user’s actual desire. 
	
	The sticky notes can be used to write down the user’s intent to share, and then are stuck to the side of the drink. They have the advantage of being cheap and usable on every bottle. However, this solution does not allow the user to change their sharing preferences without physical intervention. If the user had a no sharing note but now wants to share, the user has to physically remove the note, which will not work if the user is away from the bottle. Additionally, the notes will give no notification to the user of whether or not someone is, or has, taken the beverage. 
	
	The competitive advantage that BottleGuard has over its competitors is the ability to allow the user to express their desire, or lack thereof, to share their beverage. With the app, the user only needs to press one button to tell others that they can drink the beverage, or that they are not allowed to drink.  Being able to remotely change one’s sharing preferences and receive notifications when the bottle is accessed, in addition to BottleGuard’s competitive pricing, gives BottleGuard a high performance score and desirable location in the Price Performance chart, above the linear fit line. 
	There are two lines of linear fit, depending on the competition being taken into account.  If looking at all the competitors listed in Figure \ref{fig:kam}, including the minifridge, a linear fit line of positive slope can be drawn.  However, should the minifridge be disregarded, a negative slope linear fit line results.  This reversal of slope direction is due to the fact that the fridge, sans the price, is a very competitive solution to the unmet need.  The fridge makes sharing easy as it is as simple as opening a fridge door, and it has a strong ability to fit nearly any size beverage container.  However, when pricing is taken into account, the minifridge becomes much less viable.  Therefore, should one consider that the fridge is no longer a competitor, a negatively sloped line is drawn due to the fact that sticky notes are extremely cheap and easy to use.
	
		\addcontentsline{toc}{subsection}{Key Attributes}
	\subsection*{Key Attributes}
	The unmet need story highlights the ability to share the beverage whenever the user wants to without being physically present, therefore \textbf{ease of sharing} is the most important key attribute. Without this attribute, the user is unable to express their desire to share the beverage or not. If the user is willing to share but did not tell anyone, a person who wants to take the beverage will have to ask for permission first, or will not take the drink assuming no permission is needed, which could be inconvenient for the user. But if the user does not want to share and did not notify others, those others could take the beverage assuming permission is given. Next, \textbf{access notification}, defined as how quickly the user  is notified of unwanted access, is an important attribute to look for in the product. Without access notification, the user would not know when someone is taking beverage without permission. Following access notification comes \textbf{universal fit}. This attribute allows the user to use the product on a variety of bottles with different cap sizes, which allows the user to enjoy different beverages without having to buy different products for different bottles. Finally, \textbf{compactness} allows the user to use the product without taking up too much space in the fridge or any place where the user keeps the bottle.
	
		\addcontentsline{toc}{subsection}{Price}
	\subsection*{Price}
	The prices shown in the figures above are essentially all one time purchase calculations.  All competitors are a one time purchase that do not require any part replacements. The only exception is for the sticky notes pack, which has to be bought again when the pack is finished. However, due to the large amount of sticky notes that can come in one pack, the recurring price of buying new sticky notes was not calculated in the total price as one could probably get away with buying about one pack a year. Since BottleGuard’s batteries are rechargeable, there is no need to buy extra batteries or replacements for the device. Therefore giving BottleGuard a one time purchase price as well.  
	
	\addcontentsline{toc}{section}{Product Overview}
	\section*{Product Overview}
		\addcontentsline{toc}{subsection}{Usage}
	\subsection*{Usage}
	The product is meant to replace the existing cap of a bottle and screws onto the top of the bottle itself. Once attached, the cap is activated through the app on the consumer’s mobile device using a Wi-Fi connection. The LED will indicate the sharing mode chosen by the consumer (Green means go, Red means no). If the BottleGuard is unscrewed during the no sharing mode, the consumer will receive an app notification which includes a time stamp and a sound from the speaker will alert any person in the vicinity.
	
	\addcontentsline{toc}{subsection}{Technical Details}
	\subsection*{Technical Details}
	\begin{figure}[H]
		\centering
		\includegraphics[width=10cm]{design.png}
		\label{fig:des}
		\caption{Side view of BottleGuard. All components except the Micro-USB cable are visible.}
	\end{figure}
	The BottleGuard features the following components:
	\begin{itemize}
		\item 	Circular PCB - holds all components
		\item 	Wifi/Microcontroller chip - sends Wi-Fi notification
		\item 	Gyroscope Module - checks to make sure there is no turning of the cap (checks for rotation every 0.5 secs)
		\item 	RGB LED - blinks red or green for 0.1 seconds every second
		\item 	Speaker - makes a noise when BottleGuard is moved if sharing is turned off
		\item 	2 Coin Cell Batteries \& Battery Holder - rechargeable and powers unit for 10 days of average use (drink opened 3 times per day)
		\item 	Micro-USB port - used to charge the batteries
	\end{itemize}
	\addcontentsline{toc}{subsection}{Access Detection and Notification}
	\subsection*{Access Detection and Notification}
	The gyroscope on the unit is built to detect movement in all 3 axes as well as object orientation. Initial setup will involve placing the BottleGuard on a level surface and calibrating the sensor by turning it in both directions at varying speeds. After initial setup, the device is ready to be used. 
	
	If the device is set to “no sharing” the gyroscope will check for rotation of the device every 0.5 seconds. The gyroscope does this by taking a current base value once activated and this value only changes if the device is rotated (as in the cap is opened from the bottle). All values are sent to the microcontroller to be evaluated by pre-programmed code. If there is a change detected, the microcontroller will detect this value change and send signals to the Wi-fi card as well as the speaker. The Wi-fi card will temporarily become active to send a wifi notification to the user’s phone through a web service cloud and the speaker will sound an alarm for 3 seconds. 
	
	\addcontentsline{toc}{subsection}{Initial Wi-Fi Setup and Notification Delivery}
	\subsection*{Initial Wi-Fi Setup and Notification Delivery}
	The initial setup of the device mainly involves using a mobile device to give the wifi chip preliminary data to connect to the home network. A similar method is observed in Google’s Home Mini speaker where the device itself will create an access point which allows the user to connect their phone to. Once connected, the user can copy over their home wifi settings and the device should be able to connect to the home wifi network. Likewise, the wifi chip on the BottleGuard is able to switch between receiving and transmitting which allows it to pair to the user’s phone and then connect to the home network in the same way the Google Home Mini does. When the app is opened, it will prompt to add a new device. When the BottleGuard is not setup yet, it will be programmed to default to a receiving state so that the consumer can connect to it using Wi-Fi and so that the app can copy over the wifi settings from the user’s phone (the app will prompt which network details to copy over to the BottleGuard). Once successful, the app will do some preliminary testing by asking the user to twist the cap multiple times to make sure notifications are working.
	
	\addcontentsline{toc}{subsection}{Mobile Application}
	\subsection*{Mobile Application}
	The app notification system utilizes an IOT method similar to other devices such as Nest thermostats and wifi humidity sensors but on a much simpler level. In context, IOT devices do not bear a large amount of programming or complex code and lean more on servers to do all task management and data processing. For example, a wifi thermostat only sends temperature information to the server and does no other calculation or process. Likewise, BottleGuard’s wifi chip only serves to send one kind of signal which is only when the bottle is opened and does not send any other data in order to save battery. Once the signal is sent to the server, it sends data to the app to notify the consumer on their mobile device. The data from the cloud only includes a simple time and date of the access to the beverage. 
	
	The app does not do any data processing and mostly serves to format any data in a way that is easy to read. The data is obtained from the server which provides the time and date data when the cap is opened. The app has 3 main sections for use with the BottleGuard:
	
	\begin{enumerate}
		\item \textbf{Initial Device Setup} - This section involves the Wi-Fi and Gyroscope setups described in the previous sections and is used to setup the BottleGuard for the first time.
		\item \textbf{Main Menu} - The main menu of the app has two buttons which are pressed by the user to indicate sharing or no sharing. Only one button can be pressed at a time It also shows the most recent time the BottleGuard was moved in a “no sharing” state. 
		\item \textbf{History} - This section shows a list of the times that the BottleGuard was opened.
	\end{enumerate}
	
	\addcontentsline{toc}{subsection}{Universal Fitting Cap}
	\subsection*{Universal Fitting Cap}
	BottleGuard features a universal fitting cap that is able to fit on the two most common bottle cap size: 28mm and 38mm diameter, found in soda bottles and milk bottles respectively. Noticing that the inner diameter of bigger caps (total diameter minus the threading) is exactly the same size as the smaller cap, BottleGuard utilizes a molded cap that incorporates both of these caps on the same mold. This design consists of a large 38mm diameter cap for milk bottles, which then contains the threading of the smaller cap within its empty middle section. When used on a large bottle, the smaller cap will fit snugly into the opening, and when used on a small bottle, the larger cap will stick out of the bottle neck.
	
	Similar products have achieved this such as the MagicTap which is a hand pump designed to fit on most bottles. Our design is leveraged from this concept - refer to the Appendix for pictures of the cap design.
	
	\addcontentsline{toc}{section}{Program Risk and Mitigation}
	\section*{Program Risk and Mitigation}
	In terms of consumer reception risk, there is a possibility that consumers will opt for the cheaper sticky notes or KwikTop instead. To mitigate this risk, we will market BottleGuard’s ease of sharing and ability to notify, which are attributes that are uncharacteristic of other products in its price vicinity. Other than that, we believe our asking price of \$25 should not significantly deter consumers from purchasing the product for any other reasons.
	
	In terms of failure modes, one possible complication is the development of mold or other bacteria on the device as a result of the user failing to properly wash it. To illustrate, the mold could come in contact with the bottle contents. When the user drinks the bottle contents, he/she becomes unknowingly poisoned. This can lead to severe upset stomachs and nausea/vomiting. To mitigate the risk of this scenario occurring, we will add a clear warning in the instruction manual stating that the device needs to be regularly washed to avoid this issue. 
	
	A second possible complication is that the small size of the device makes it a choking hazard. This primarily applies to young children, who may easily access the device, attempt to eat it, and choke. This could have fatal consequences. In order to mitigate this risk, we will add a choking hazard warning label, a warning directed at parents that detail it is unsafe to leave unsupervised children with the device, and an age restriction to further deter unsupervised children from playing with device.
	
	Lastly, a possible complication may arise from heating the device to sufficiently high temperatures. Our device is not engineered to withstand high heat sources such as microwaves and ovens. In these cases, heated batteries may pose serious harm if they release flammable substances and are ignited, creating an explosion. To mitigate this risk, we will include a warning label that directs users to not use the device in microwaves, ovens, or any other areas with high heat.
	
	
	
	
		\addcontentsline{toc}{section}{Financial Summary}
	\section*{Financial Summary}
		\addcontentsline{toc}{subsection}{Key Milestones}
	\subsection*{Key Milestones}
	\begin{itemize}
		\item 	Unit cost: \$4.069
		\item Unit retail price: \$25
		\item Development cost: \$274.4K
		\item Break even quantity: 13107
	\end{itemize}

	
	\addcontentsline{toc}{subsection}{Unit Cost Breakdown}
	\subsection*{Unit Cost Breakdown}
		\begin{figure}[H]
		\centering
		\includegraphics[width=10cm ]{DesignLabeled.png}
		\label{fig:design}
		\caption{Side view of BottleGuard with all parts labeled with their ID from the BOM below.}
	\end{figure}
\begin{table}[H]
	\includegraphics[width=\textwidth]{bom_small.png}
	\label{fig:bom}
	\caption{Detailed Bill of Materials for each part in BottleGuard. BOM with MOQ and part numbers/links to source vendors is included in the appendix..}
\end{table}
	\addcontentsline{toc}{subsection}{Development Cost Breakdown}
	\subsection*{Development Cost Breakdown}
	\subsubsection*{Labor: \$258K}
		\begin{table}[H]
		\includegraphics[width=\textwidth]{wbs_small.png}

		\caption{Estimated work weeks required for the development of BottleGuard. Detailed WBS with salary calculations as well as manager / benefits calculations is included in the appendix.}
	\end{table}
From the work weeks in Table 2 and standard industry salaries, the total salary needed for engineers is \$127K. With department managers and project managers each needing 25\% of total engineering time, we would need an additional \$44K for department managers and \$35.3K for project manager. In total, it would take \$206.4K for all labor costs, which comes up to \$258K with 25\% benefits.
	\subsubsection*{Capital: \$5K}
Since BottleGuard’s outer design is relatively simple, molds and tooling for the enclosure and the cap would cost an estimated \$5K total.
	\subsubsection*{NRE: \$10K}
	The universal cap design is the most important part of the design, and thus an extra \$10K is allocated to the mechanical engineers to help expedite the process. This money will go towards buying required equipment for testing as well as contracting any external help if needed.
	
	\subsubsection*{Risk Inventory: \$2.7K}
	As BottleGuard’s break even quantity (13107) is way above the MOQ of almost all components, we can keep a low risk inventory on all of them. One notable exception is the gyroscope, which must be purchased in reels of 5000. With the break even quantity of 13107, we would have to order 15000 gyroscopes at \$1.44, leaving us with 1893 gyroscopes as risk inventory. The total cost for this risk inventory is \$2.7K. 
	
	\addcontentsline{toc}{subsection}{Break Even Quantity Analysis}
	\subsection*{Break Even Quantity Analysis}
	Each unit is being sold for a retail price of \$25 for a profit of \$21 per device. The total development cost including labor, capital, NRE, and risk inventory is \$274.4K, giving us:
	\[ \text{Break Even Quantity} = \frac{\text{Total Development Cost}}{\text{Profit per unit}} = 13107 \] 
	The complete calculation including the initial break even calculation as well as the first pass to reduce MOQ is included in the appendix.
	
		\addcontentsline{toc}{section}{Delivery Date and Key Milestones}
	\section*{Delivery Date and Key Milestones}
Our three engineering teams (Software, Electrical, and Mechanical) will work towards completing our main deliverables, which are the app, electronics, plastic cap, and the final product after integration and verification. The chart below shows the complete breakdown of the weeks needed per deliverable as well as what each team is responsible for. The software team will be working on the app, which we estimate to take 5 weeks. The plastic cap will be completed by the mechanical engineering team at the end of 6 weeks. The electrical engineering will design and wire the electronic components, while the software team will code the microcontroller. The electronic components for the device will be ready by the end of week 8. After 4 weeks of integration and verification by all teams, the prototype will be completed. We estimate the full product design to take around 16 weeks, and during this time period we will continually improve our design and try to make the cycle more efficient. Accounting for the time needed to set up mass production, marketing, and some other operations, we expect our product to enter the market within a year. 
\begin{figure}[H]
	\includegraphics[width=\textwidth]{milestones.png}
	\label{fig:milestones}
	\caption{Complete product development cycle with key milestones marked}
\end{figure}
	\addcontentsline{toc}{section}{Summary and Recommendations}
	\section*{Summary and Recommendations}
		
		Given current rising housing costs, most young adults live with other roommates or friends in situations that usually involve sharing a refrigerator. Some of them, like in Bailey’s case, want to share their beverages. However, they also wish to control the sharing process and dictate when they want to share. As previously illustrated, \textbf{BottleGuard} is the best fit to address these needs. BottleGuard stands strong amongst other competitors by enabling its customers to display their intentions and share their drinks with minimal fuss.  Through a simple app, the customer can control the red or green light shown on top of the device, without the need of physical attendance. Most importantly, it provides them flexibility — the ability to clearly communicate their sharing intent through a simple touch of a button on a phone app.
		
		Moreover, BottleGuard relays accurate and timely notifications through the app to the customer when someone tries to open the cap without permission (or when the bottle is on “not sharing” mode). As proven earlier, the device can withstand normal movements in the fridge without giving off false notifications due to our efficient utilization of the gyroscope. In addition, thanks to the alarm sound triggering, the person accessing the drink is further reminded of the owner’s desire not to share. Lastly, as illustrated in the previous sections, our product fits most standard caps and is compact so it does not take much space in the fridge. 
		
		Studies show that almost 79 million U.S. adults are living in a shared household (Fry, 2018), illustrating that there exists a large market for BottleGuard to penetrate. Moreover, the device will also help employees working in offices that contain shared refrigerators, further increasing our number of targeted customers. Designed to meet customers’ needs, BottleGuard will certainly be a leading product in the market. Taking all of these factors into consideration, it is evident that BottleGuard will be financially viable. With an investment of \$275,000, you can help bring BottleGuard to reality and be the first to tap into an unexplored market.
		

		
		\newpage
		\addcontentsline{toc}{section}{Appendix}
		\section*{Appendix}
			\addcontentsline{toc}{subsection}{Unmet Need}
		\subsection*{Unmet Need}
		Bailey is a college student who lives in an apartment with several roommates. They all share the same fridge, and some of the items within. Although she is generally fine sharing her drinks with others, sometimes Bailey would like to keep them to herself when she gets a soda that is difficult to find or when the drink is running out. Due to their busy schedules, however, she finds it difficult to communicate that with all of her roommates. She resents the fact that sometimes her roommates help themselves to her drinks without asking for permission first because they were not aware that she didn't want to share. Bailey would like a way to remind her roommates in the moment of whether or not she is willing to share. Additionally, she wants to know when someone tries to open the bottle and take her soda when she has not granted permission.
		
		She has tried different methods of expressing her intentions, such as sticking Post-it notes on her bottles or sending a text message to their roommate group that she'd prefer if they didn't take her drink. However, these methods have not proven to be effective since text messages are often ignored or overlooked, and Post-it notes are prone to falling off the bottle. She finds a bottle lock called KwikTop but decides against using it since she understands that everyone’s schedule is different from hers, and such an inflexible method of keeping the drink to herself defeats the idea of sharing the drink, and can sour Bailey’s relationship with her roommates. Moreover, a bottle lock comes at a fixed size and can only fit on one type of bottle, while she wishes to share a variety of drinks that come in markedly different bottle sizes like soda bottles or milk jugs. Ideally, she also wants something that does not take up much room in the fridge, as she only has a limited amount of fridge space allocated to her.
		
		Our device, BottleGuard, will solve all of Bailey’s problems. BottleGuard is a compact cap replacement that can be fitted on to various types of bottles. It would let Bailey’s roommates know if she wanted to share the drink through a sharing mode that displays a green light to indicate a desire to share and a red light to indicate when one does not want to share. With a simple button press from the app, Bailey can set the light on the device from red to green from anywhere in the world with an Internet connection. The green light would then indicate her consent to share for the next couple minutes, allowing her roommate to pour their drink, before changing back to red again and signifying that permission is needed.  Bailey can also choose to keep the light green and leave sharing open. When the light is red, any attempt to open the bottle will trigger a warning sound and send a notification with the time the bottle was opened to Bailey’s phone. Bailey can now be more secure in knowing that she can be aware of if and when her roommates try to take her drinks, or use them without permission.
		
		\addcontentsline{toc}{subsection}{Concept of Operations}
		\subsection*{Concept of Operations}
		
		As a potential customer, I would like to purchase the device at a department store or through an online platform like Amazon. Once received, I expect a straightforward instruction manual that will allow me to configure my device quickly and tell me to recharge the device using the included Micro-USB cable before first use. I could then proceed to replace the bottle’s cap with the device. I expect that my device should be able to fit an assortment of bottles, and not take up a considerable amount of space in the fridge. 
		
		After the device is prepared, I could download a free app on my smartphone that will interface with the device. First, I expect the app to have one buttons for “not sharing” and “sharing” modes, which I can toggle between depending on whether or not I wished to share the drink. When I select “not sharing”, the device should flash a red light. When I select “sharing”, the app should give me the option to "share" for a limited amount of time or leave the device on "sharing" mode until I switch it back. Second, I expect there to be a notification settings option that will allow me to choose to enable all notifications, enable notifications only when device is on not sharing mode, or to disable all notifications when it is on sharing mode. Lastly, I could connect my phone to my device and connect the device to my home WiFi network through a menu in the app. 
		
		Once configured, I should still be able to open my beverage normally and receive a notification about it. If I leave the house and travel anywhere in the world, I could receive a notification via WiFi whenever my beverage is opened. Simultaneously, the device’s speaker will make a warning sound. I should receive these notifications promptly so that I can immediately message my roommates and make an inquiry about their use of my drink. I should receive a timestamp that will let me know when my drink was accessed, so that I can more easily find out who opened my drink. I expect these timestamps of each use (along with the sharing mode status at the time) to appear in an organized access log on the app sorted by time. 
		
		When I have finished my beverage, I could simply remove the device from the old beverage bottle and attach it to a new one without any configuration. My only responsibility should be to recharge the device every week through the micro-USB charging port. When the device is low on power, I expect it to send me a notification through the app to remind me of charging.
		
		\addcontentsline{toc}{subsection}{Complete Specifications}
		\subsection*{Complete Specifications}

		
		\begin{figure}[H]
		\centering
		\includegraphics[width=\textwidth]{specs.png}
		\end{figure}
		\setlength{\parindent}{0pt}
		\addcontentsline{toc}{subsection}{Proof of Concept}
		\subsection*{Proof of Concept}
		\subsubsection*{Power Consumption Calculations}
		\begin{itemize}
			\item Sensor - 1.71 V and 0.85mA
			\item Speaker - 1.5 V and 2mA (transmitting only)
			\item Wifi - 2.5 V and 70mA (transmitting only)
			\item LED - 1.8V and 15 mA (never on during transmitting)
			\item Battery - 3V and 620 mAh capacity
		\end{itemize}

		\textbf{Totals}:
		\begin{itemize}
			\item Total voltage is = 5.71 V = 2 batteries needed (3 V each)
			\item Total Current is = 0.87mA idle / 70mA transmitting for 10s + 2mA sound for 3s
			\item Total mAs (milli-Amp seconds) - 210 mAh x 60 mins x 60 secs x 2 batteries = 1,512,000 mAs 
		\end{itemize}

		
		\textbf{Daily base consumption}
		
		Gyroscope will update position data every 0.5 seconds (172,800 times a day) for 0.1s:
		\begin{center}
			0.85 mA x 0.1s x 172,800 times daily = 14,688 mAs daily
		\end{center}
		
		LED will turn on every second for 0.1 second which makes a blinking visual:
		\begin{center}
			15 mA x 0.1s x 86,400 times daily = 129,600 mAs daily
		\end{center}
		
		\textbf{Bottle Opened Consumption}
		
		During transmission (when bottle is opened), wifi chip will transmit for 10 seconds and speaker will sound for 3s:
		\begin{center}
			70mA x 10s = 700 mAs (WiFi)
		2mA x 3s = 6 mAs (Speaker)
		\end{center}
	
		Assuming the bottle is opened 2 - 3 times a day, the average daily consumption would be:
		\begin{center}
			14,688 mAs + 129,600 mAs +  (700 mAs x 3 times) + (6 mAs x 3 times) = 146,388 mAs daily
		\end{center}
		
		This means the battery will last at most:
		\begin{center}
			1,512,000 mAs total/ 146,388  mAs daily = 10.18 days per charge
		\end{center}
		
		\subsubsection*{Gyroscope Tolerance Threshold}
		The following test checks to make sure it is practically impossible to perfectly take off and replace the cap within the span of 0.5 seconds.
		
		\textbf{Methods}: Two standard bottles were tested to accurately capture real world objects that our product would be applied to. The first was a standard one gallon milk jug, filled $\frac{4}{5}$ths with water. The second was a standard 20 oz bottle, filled half-way through with water. In order for a bypass attempt to be successful, the bottle cap after opening must preserve its original orientation. 
		
		We performed two experiments to study the plausibility of a successful bypass attempt. In our first experiment, we sought to explore if it was possible to open a bottle in less than half a second. To do this, we attempted to open the bottles with tightly closed caps as fast as we could, with no regard for returning the bottle cap to its original orientation. Four trials were performed for each type of bottle. 
		
		In our second experiment, we sought to define whether or not bottles could be opened in less than half a second and returned to their original orientations so that the gyroscopes would not register any change. We used a similar approach as in our previous experiment, but took accuracy into account this time. Here, three trials were performed for each type of bottle. 
		
		\textbf{Results: }
		\begin{table}[H]
			\centering
			\begin{tabular}{|c|c|}
				\hline
				\textbf{Milk bottle (seconds)} & \textbf{Water bottle (seconds) } \\ \hline
				0.51 &1.0 \\ \hline
				0.52 & 1.36\\ \hline
				0.52&1.34\\ \hline
				0.90&1.20\\ \hline
			\end{tabular}

			\caption{Trial attempts to open bottle as fast as possible, without any regard for orientation.}
			\label{trial1}
		\end{table}
				\begin{table}[H]
			\centering
			\begin{tabular}{|c|c|}
				\hline
				\textbf{Milk bottle (seconds)} & \textbf{Water bottle (seconds) } \\ \hline
				2.44&
				2.40\\ \hline
				2.13&
				2.24\\ \hline
				1.45&
				2.14\\ \hline
			\end{tabular}

			\caption{Trial attempts to open bottle as fast as possible while also returning it to the original position.}
					\label{table:trial2}
		\end{table}

		\textbf{Analysis:} Our results indicate that even if accuracy is disregarded, it is likely not possible to open the bottle cap without triggering the notification (Table \ref{trial1}). While 3 of the 4 trials for the milk bottle was opened just a second or two more than the limit of 0.50 seconds and suggest the possibility that milk bottles could be opened in less than 0.50 seconds, our gyroscope design accounts for this potential complication due to its ability to detect changes in orientation. In our second experiment, we found that it took significantly longer to open the bottle and take accuracy into account (Table \ref{table:trial2}). Given that the fastest time was nearly three times more than the limit of 0.50 seconds, this offers strong evidence to support that it is near impossible to open the cap without triggering the sensor. While there is variability across our data, values that seem to be positive outliers can largely be attributed to human error.
		
		\subsubsection*{Cap Design (leverage)}
		\begin{figure}[H]
			\centering
			\includegraphics{magictap.png}
			\label{fig:tap}
			\caption{Cap Design of Magic Tap. Our cap design will be very similar, except the two caps are stacked onto each other}
		\end{figure}
		\addcontentsline{toc}{subsection}{Complete FMECA}
		\subsection*{Complete FMECA}

		\begin{tabularx}{\textwidth}{ | X | X | X | >{\small}c | >{\small}c | >{\small}c | >{\small}c | X | }
			\hline
			& \textbf{Cause} & \textbf{Effect} & \rotatebox{90}{\textbf{Severity}} & \rotatebox{90}{\textbf{Probability}} & \rotatebox{90}{\textbf{Detectability}} & \textbf{RPN} & \textbf{Mitigation}   \\ \hline
			Cap part of device can come into direct contact with liquid
			& Device is never washed & Mold forms due to improper cleaning, customer gets poisoned & 5.0 & 4.0 & 8.0 & 160 & Warning in manual to properly wash device \\ \hline
			Device is not heat resistant & Device is put in oven  & Device's enclosure melts down, batteries explode, causing a fire that burns down their house and causing injuries
			& 7.0 & 2.0 & 8.0 & 112 & Warning in manual to avoid placing device in oven \\ \hline
			Device contains metal & Device is put in microwave & Device's batteries and metal components spark and potentially explode and cause a fire & 7.0 & 2.0 & 8.0 & 112 & warning in manual to avoid placing device in microwave oven   \\ \hline
			Device can roll around on the floor & Device has round edges & Customer steps on device and slips, falling to their death
			& 10.0 & 1.0 & 10.0 & 100 & Warning in manual informing users not to leave device lying on floor as it could be a slipping hazard or painful to step on  \\ \hline
			Device is easily accessible to small children & Child in house opens the fridge and chew on the device & Child chokes on the device and dies & 10.0 & 1.0 & 8.0 & 80 & Add choking hazard warning in manual
			\\ \hline
			Device has low battery life & Battery runs out of power before the drink is finished & Cap no longer warns of attempted use but can't be recharged because it has replaced the bottle's orginical cap & 1.0 & 8.0 & 10.0 & 80 & Device sends notification to user's phone at low battery charge \\ \hline
			One or more materials are allergy-inducing & Customer exposed to device is allergic to material & Customer receives severe allergic reaction that is irritating and requires medical attention & 7.0 & 1.0 & 10.0 & 70 & Add a list of materials used in the manual as well as on the packaging of the device  \\ \hline
			Battery has no way to stop charging once it is at full capacity & Battery overcharged & Battery possibly explodes or lights on fire, injuring the customer and causing a fire & 7.0 & 1.0 & 10.0 & 70 & Modern charging circuits stop the charge when the battery is full \\ \hline
			Device has no mechanism of detecting hostile intentions & Evil roommate repeatedly take off cap to set off continuous stream of notifications & Customer is spammed with notifications & 1.0 & 5.0 & 10.0 & 50 & Customer can stop notifications from the device in the app
			 \\ \hline
			Device is thrown or swung  & Roommates get mad at customer, take out device and use it as a weapon & Customer is injured from device
			& 5.0 & 1.0 & 10.0 & 50 & Add warning against throwing device in instruction manual  \\ \hline
			Device contains access points for water & Charging port not covered
			& Customer spills water into the port, device stops working & 1.0 & 4.0 & 10.0 & 40.0 & Add a rubber cover for the charging port   \\ \hline
			Device's container is damaged & The customer exposes the device to cold temperatures (below 0C) when they try to put their devices in the freezer. & The electronics are damaged and the device no longer works
			& 1.0 & 3.0 & 10.0 & 30 & Add a warning sign against usage in cold temperatures in the user manual
			 \\ \hline
			Device circuitry is not properly insulated from outside environment & Poor manufacturing/design  & Customer spills water on device, device stops working & 1.0 & 3.0 & 10.0 & 30 & Add insulation testing during manufacture to make sure that the device is properly insulated  \\ \hline
			As IOT device, it is hacked into & IOT devices are susceptible to outside hacking & Hacker changes the code and device loses its original function. & 1.0 & 2.0 & 10.0 & 20 & Carry out pentests to make sure that the software is secure  \\ \hline
			Battery Explodes & Battery is not properly manufactured & Battery Acid melts the bottom of the cap, falling into the drink and poisons the user
			& 5.0 & 1.0 & 3.0 & 15 & Warning in manual to check device for damages \\ \hline
		\end{tabularx}
		
		\addcontentsline{toc}{subsection}{PCB Quotes}
		\subsection*{PCB Quotes}
		\begin{figure}[H]
			\centering
			\includegraphics[width=\textwidth]{quote1.png}
		\end{figure}
		\begin{figure}[H]
		\centering
		\includegraphics[width=\textwidth]{quote2.png}
		\caption{Quotes for PCB fabrication}
		\end{figure}
		
		\begin{figure}[H]
			\centering
			\includegraphics[width=\textwidth]{quote3.png}
			\caption{Quote for PCB assembly}
		\end{figure}
		\addcontentsline{toc}{subsection}{Detailed Work Breakdown Structure}
		\subsection*{Detailed Work Breakdown Structure}
		\begin{table}[H]
			\centering
			\includegraphics[width=\textwidth]{wbs_large3.png}
			\caption{ Detailed WBS including all detailed salary calculations and manager hours}
		\end{table}
		\addcontentsline{toc}{subsection}{Detailed Bill of Materials}
		\subsection*{Detailed Bill of Materials}
		\begin{table}[H]
			\includegraphics[width=\textwidth]{bom_large.png}
			
			\caption{Detailed BOM including product number or links to the source vendors. Mouser refers to Mouser.com.}
		\end{table}
		\addcontentsline{toc}{subsection}{Break Even Calculations}
		\subsection*{Break Even Calculations}
				\begin{table}[H]
			\includegraphics[width=\textwidth]{breakeven.png}
			
			\caption{ Break even calculations including initial and first pass}
		\end{table}
		\addcontentsline{toc}{subsection}{Risk Inventory Calculations}
		\subsection*{Risk Inventory Calculations}
				\begin{table}[H]
			\includegraphics[width=\textwidth]{risk.png}
			\caption{Detailed risk inventory calculations}
		\end{table}
	
			
	\addcontentsline{toc}{section}{References}
	\section*{References}

	U.S. Plastic Corporation. (2003). How do I know how to calculate a cap \& neck size? Retrieved from https://www.usplastic.com/knowledgebase/article.aspx?contentkey=625. 
	
	International Society of Beverage Technologists, Thread Specs Standard. Retrieved from https://www.bevtech.org/threadspecs.asp
	
	ICOMold. (2017). Plastic Injection Mold Cost: ICOMold. Retrieved from https://icomold.com/much-injection-molding-cost/. 
	
	Fry, R. (2018). More U.S. adults now share their living space. Retrieved from https://www.pewresearch.org/fact-tank/2018/01/31/more-adults-now-share-their-living-space-driven-in-part-by-parents-living-with-their-adult-children/. 
\end{document}